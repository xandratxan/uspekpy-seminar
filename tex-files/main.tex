\documentclass{beamer}

% Title page and author information
\title{USpekPy Package}
\subtitle{Uncertainty estimation on protection quantities for x-rays using SpekPy and Monte Carlo techniques}
\author{Xandra Campo, Paz Avilés}
\institute{Ionizing Radiation Metrology Laboratory (LMRI) \newline CIEMAT, Spain}
\date{June 2024}

\begin{document}
	% Title page
	\maketitle
	% Table of contents
	\begin{frame}
		\frametitle{Table of Contents}
		\tableofcontents
	\end{frame}
	
	\section{What is USpekPy?}
	\begin{frame}
		\frametitle{What is USpekPy?}
		\begin{itemize}
			\item {\color{blue} Python package} to compute mean radiation protection quantities for a simulated x-ray spectrum with uncertainties using Monte Carlo techniques.
			\item Open source and GPLv3-licensed library compatible with Python 3. 
			\item Based on {\color{blue} SpekPy}: Python package for modelling the x-ray spectra from x-ray tubes
		\end{itemize}		
	\end{frame}
	
	\section{Main features of USpekPy}
	\begin{frame}
		\frametitle{Main features of USpekPy}
		\begin{itemize}
		\item Compute {\color{blue} mean values of radiation protection quantities} of an x-ray spectrum simulated using SpekPy: $\overline{E}$, $K_{air}$ and $\overline{h_K}$.
		\item Compute {\color{blue} mean radiation protection quantities} of an x-ray spectrum simulated using SpekPy {\color{blue} with uncertainties} using Monte Carlo techniques: first and second HVL for Al and Cu, $\overline{E}$, $K_{air}$ and $\overline{h_K}$.
		\item Perform {\color{blue} batch simulation} to compute the mean values and uncertainties of radiation protection quantities for {\color{blue} several x-ray spectra} simulated with SpekPy.
		\end{itemize}
	\end{frame}
\end{document}
